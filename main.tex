%% The first command in your LaTeX source must be the \documentclass command.
%%
%% Options:
%% twocolumn : Two column layout.
%% hf: enable header and footer.
\documentclass[
% twocolumn,
% hf,
]{ceurart}

%%
%% One can fix some overfulls
\sloppy

%%
%% Minted listings support 
%% Need pygment <http://pygments.org/> <http://pypi.python.org/pypi/Pygments>
\usepackage{listings}
%% auto break lines
\lstset{breaklines=true}

\usepackage{svg}
\usepackage{graphicx}
\usepackage{sepfootnotes}
\usepackage{csquotes}
\usepackage{subfig}
\usepackage{listings}
\usepackage{xcolor}
\usepackage{hhline}

\input{reviewing}

\lstset{
    basicstyle=\ttfamily\footnotesize,
    backgroundcolor=\color{gray!10},
    numbers=left,
    numberstyle=\tiny\color{gray},
    stepnumber=1,
    numbersep=5pt,
    showstringspaces=false,
    breaklines=true,
    frame=single
}

%%
%% end of the preamble, start of the body of the document source.
\begin{document}

%%
%% Rights management information.
%% CC-BY is default license.
\copyrightyear{2024}
\copyrightclause{Copyright for this paper by its authors.
  Use permitted under Creative Commons License Attribution 4.0
  International (CC BY 4.0).}

%%
%% This command is for the conference information
\conference{The ISWC 2024 Posters and Demos Track}

%%
%% The "title" command
\title{Optimizing source selection in link traversal queries for sensor data using a rule-based approach}


%%
%% The "author" command and its associated commands are used to define
%% the authors and their affiliations.
\author[1]{Bryan-Elliott Tam}[%
email=bryanelliott.tam@ugent.be,
orcid=0000-0003-3467-9755
]
\cormark[1]

\author[1]{Ruben Taelman}[%
orcid=0000-0001-5118-256X,
email=ruben.taelman@ugent.be,
url=https://www.rubensworks.net,
]
\author[1]{Pieter Colpaert}[%
orcid=0000-0001-6917-2167,
email=pieter.colpaert@ugent.be,
url=https://pietercolpaert.be
]

\cortext[1]{Corresponding author.}

\address[1]{IDLab,
Department of Electronics and Information Systems, Ghent University – imec}

%%
%% Keywords. The author(s) should pick words that accurately describe
%% the work being presented. Separate the keywords with commas.
\begin{keywords}
  Linked data \sep
  Link Traversal Query Processing \sep
  Fragmented database \sep
  Descentralized environments
\end{keywords}

%%
%% This command processes the author and affiliation and title
%% information and builds the first part of the formatted document.
\maketitle

\begin{abstract}
% <!-- Context -->
Link Traversal queries face challenges in source selections and completeness due to the size of the web.
Reachability criteria define both completeness and source selection policies, based on the traversal of links encountered by query engines.
However, the size of the search domain remains the bottleneck of the approach.
% <!-- need -->
Web environments often have structures exploitable by query engines to perform precise source selection.
Current criteria rely on using information from the query definition and regular expressions to select sources.
However, it is difficult to use them to traverse environments where logical expressions indicate the location of resources.
For example, the traversal of interlinked documents respecting expressions such as 
``the data produced after the first of September are stored at \texttt{ex:afterFirstSeptember}'' 
are difficult to express using current reachability criteria.
% <!-- task -->
We propose to use a rule-based reachability criterion that captures logical statements expressed in hypermedia descriptions 
within linked data documents to define the source selection policy of traversal queries.
%<!-- object -->
In this poster paper, we show how the Comunica Web querying engine is modified to, during link traversal, 
take hints from a hypermedia control vocabulary, to more efficiently query over subsets of a sensor dataset published 
as linked data documents with semantic hypermedia controls.
% <!-- findings -->
Our preliminary findings show that using this strategy, the query engine can significantly reduce the number of HTTP requests 
and the query execution time without sacrificing the completeness of results 
when executing queries aligned with the available hypermedia controls.
% <!-- conclusion -->
Given the promising result of this initial approach, we will extend our implementation beyond a time-based criterion, 
to support general reasoning over linked data in the source selection process of traversal queries.
\end{abstract}

\section{Introduction}

\href{https://lod-cloud.net/#diagram}{The increasing amount of available Linked Data on the Web},
prompts the need for efficient query interfaces, with SPARQL endpoints as the most prominent one.
During a typical SPARQL query execution, the interface takes the whole query load and delivers the results to the client.
This may cause high workloads and is partially the reason for the historically low
availability levels~\cite{aranda2013}.
Academics have made efforts to introduce alternative Linked Data publication methods
that enable clients participating in the query execution process.
The goal is to lower server-side workloads while keeping fast query execution to the client~\cite{Azzam2021}.
The TREE hypermedia specification is an effort in that direction~\cite{ColpaertMaterializedTREE, lancker2021LDS},
that introduces the concept of domain-related fragmentation of large RDF datasets.
For example, in the case of periodic measurements of sensor data,
the fragmentation can be made based on the publication date of each data entity.
TREE aims to fragment datasets in a way that enables clients to easily fetch subsets that fully answer a given query.
The data inside a fragment are bounded with constraints that are expressed using hypermedia descriptions~\cite{thomasFieldingPhdThesis}.
More precisely, each fragment declaratively describes the constraints of the data it contains, and links to other fragments.
Since TREE fragments are hyperlinked Linked Data documents,
clients must traverse over these documents to find data,
which makes Link Traversal Query Processing (LTQP)~\cite{Hartig2016} a suitable technique for answering SPARQL queries over it.

LTQP typically starts with a set of seed URLs that are dereferenced.
From these dereferenced documents, links to other documents are dereferenced recursively.
So far, applications on top of TREE datasets~\cite{ColpaertMaterializedTREE, lancker2021LDS}
have resorted to custom traversal implementations to find data matching specific data needs.
Our aim in this work is to explore how to execute generic SPARQL queries over TREE datasets through LTQP.
We do not aim to extend the existing TREE specification,
but merely to introduce LTQP-specific link pruning techniques that exploit the structural properties of TREE,
similar to the exploitation of structural properties when performing LTQP over the Solid environment\cite{taelman2023}.
Specifically, we will make use of the SPARQL FILTER expression to prune links that match the constraints of TREE fragments.

As a running example throughout this paper, we consider the publication of sensor data.
For example, the query the listing \ref{lst:system} targets the DAHCC~\cite{dahcc_resource} dataset.
We have created queries to get the measures between a specific time interval (the filter expression will vary in our experiment) 
and information about the sensor using a metadata file, which is available at listing \ref{lst:system}.

\begin{figure}[h]
    \begin{minipage}{0.50\textwidth}
        \centering
        \lstinputlisting[language=,frame=single]{code/example_sparql_query.ttl}
    \end{minipage}
    \hspace{0.05\textwidth}
    \begin{minipage}{0.40\textwidth}
        \centering
        \lstinputlisting[language=,frame=single]{code/example_tree_relation.ttl}
    \end{minipage}
    \caption{On the left, is a SPARQL query to get sensor measurements and metadata of a fragmented dataset.
    On the right, is the hypermedia description of the fragment following the TREE specification, the next fragment
    has a constraint on publication time $?t>= \text{2022-01-03T09:47:59.000000}$.}
        \label{lst:system}
\end{figure}
\section{Pruning links using filter expressions}

\sepfootnotecontent{sf:implementation}{
\href{https://github.com/constraintAutomaton/comunica-feature-link-traversal/tree/feature/time-filtering-tree-sparqlee-implementation}{https://github.com/constraintAutomaton/comunica-feature-link-traversal/tree/feature/time-filtering-tree-sparqlee-implementation}.  
}

\sepfootnotecontent{sf:queries}{
\href{https://github.com/TREEcg/TREE-Guided-Link-Traversal-Query-Processing-Evaluation/tree/main/evaluation/query}{https://github.com/TREEcg/TREE-Guided-Link-Traversal-Query-Processing-Evaluation/tree/main/evaluation/query}
}

Most research around LTQP centered around query execution Linked Open Data.
Given the pseudo-infinite number of documents on the Web, traversing over all documents is practically infeasible.
To define completeness, different reachability criteria~\cite{hartig2012} were introduced that allow to discriminate the links followed from the internal data store of the engine.
Recently, an alternative direction was introduced where the query engine uses the structure from the data publisher to guide itself towards relevant data sources~\cite{taelman2023, verborgh2020}.

We define our approach as a rule-based reachability criterion.
Our approach builds upon the concept of guided link traversal, and structural assumptions~\cite{taelman2023} to exploit the structural properties of TREE datasets to prune irrelevant links.
Concretely, we interpret the hypermedia descriptions of constraints in TREE fragments as boolean equations as shown in Figure~\ref{lst:system}.
Upon discovery of a document, the query engine gathers the relevant triples to form the boolean expression of the constraint of the next fragment.
The operators of the expressions are inferred by mapping the triple from the document and the definition of the TREE specification.
After the parsing of the expression, the filter expression of the SPARQL query is pushdown into the engine's dereferencing component.
Lastly, the two boolean expressions are evaluated to determine if they can be satisfied, upon satisfaction the IRI targeting the next fragment is added to the link queue otherwise the IRI is pruned.

We have implemented our approach in a fork of the LTQP version of Comunica~\cite{comunica}~\sepfootnote{sf:implementation}.

\subsection{Experimental results}

To evaluate our approach, we executed four queries similar to the one in figure \ref{lst:system}.
All queries are available online~\sepfootnote{sf:queries}.
They were executed over the DAHCC participant 31 dataset (487 MB) with a timeout of two minutes.
We fragmented the (same) dataset following the TREE specification with a one-ary tree topology with 100 and 1000 nodes.


\begin{table}[ht]
    \centering
    \begin{tabular}{|c|c|c|c|c|c|c|c|}
        \hline
        \textbf{Fragments} & \textbf{Query} & \textbf{Time (ms)}  & \textbf{Time-rule (ms)} & \textbf{Req-rule} & \textbf{Res-rule} \\
        \hline
        100 & Q1 & x & 8,892& 3 & 0 \\
        100 & Q2 & x & 3,541& 3 & 1 \\
        100 & Q3 & x & 59,274& 8 & 8,166 \\
        \hhline{|=|=|=|=|=|=|=|=|}
        1000 & Q1 & x & 1,171& 3 & 0 \\
        1000 & Q2 & x & 734& 3 & 1 \\
        1000 & Q3 & x & 39,987& 51 & 8,166 \\
        \hline
    \end{tabular}
    \caption{
    The predicate-based reachability criterion is not able perform the queries. 
    With the  rule-based criterion perform better with a larger number of fragments.
    X means that the query did not finish before the timeout.
    Q4 is not displayed because they were not able to terminate before the timeout.}
    \label{tab:result}
\end{table}

The queries were executed with two configurations.
In the first configuration, we use a predicate-based reachability criterion where the engine follows each link of the fragmented dataset.
For the second one, we use our rule-based reachability criterion approach.
As shown in Table \ref{tab:result} no query could be answered below the timeout value by following every fragment.
The reason is that the bottleneck of LTQP is the large number of HTTP requests~\cite{Hartig2016} and this approach dereferences a large number of non-contributing data sources.
For the configurations with our rule-based reachability criterion, we see that the queries with 1000 fragments perform better than
the ones with 100 fragments, particularly when the query has one or zero results.
In those cases, the query execution time is almost one order of magnitude lower with the largest number of fragments.
With Q3 we can see that the percentage of reduction is 32\%, this lowering of performance might be caused by the observed increase by a factor of 17 in HTTP requests.
This raises an interesting observation because we do not observe a reduction of execution time proportional to the reduction of HTTP request.
We did not affect the query plan, thus the conclusion of~\citeauthor{taelman2023} stating that the query plan can be the bottleneck of the approach might not be the only explanation.
The size of the internal data source may have a bigger impact on the performance than noted in previous studies.
The query Q4 was not able to be answered, with both fragmentations, because it covers a far larger range of publication date and hence requires more data to be downloaded and more processing time.

\input{section/Conclusion}

% --- Bibliography ---

\bibliography{references}

\end{document}

%%
%% End of file
